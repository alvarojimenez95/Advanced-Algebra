\documentclass[a4paper,12pt]{article}
\usepackage[english]{babel}

\usepackage[utf8]{inputenc}
\usepackage{amssymb, amsmath, stmaryrd,  amsthm,  amsfonts, mathtools}
\usepackage{graphicx} 

\usepackage{thmtools}
\usepackage[toc,page]{appendix}
\usepackage[top=0.75in, bottom=1.25in, left=0.85in, right=0.85in]{geometry}
\usepackage{amsmath}
\usepackage{amsthm}
\usepackage{amsfonts}
\usepackage{amssymb}
\usepackage{pst-node}
\usepackage{tikz-cd}
%\usepackage[cm]{fullpage}
%\usepackage[T1]{fontenc}
\usepackage{selinput}
\usepackage{mathrsfs}
\usepackage{tikz}
\usepackage{enumitem}
\usepackage{mdframed}
\newcommand{\fcal}{\mathcal{F}}

%solution
\newenvironment{solution}
  {\renewcommand\qedsymbol{$\blacksquare$}\begin{proof}[Solution]}
  {\end{proof}}

\newenvironment{sketch}
  {\renewcommand\qedsymbol{$\square$}\begin{proof}[Sketch of proof]}
  {\end{proof}}

\newcommand{\mo}{\mathrm{Mod}}
\newcommand{\modul}{\, \text{mod}\,}
\newcommand{\theo}{\mathrm{Th}}
\newcommand{\m}{\mathcal{M}}
\newcommand{\acal}{\mathcal{A}}
\newcommand{\ccal}{\mathcal{C}}
\newcommand*{\threeemdash}{\rule[0.5ex]{3em}{0.55pt}}
\newcommand{\Hom}{\mathrm{Hom}}
\newcommand{\ide}{\mathrm{id}}
\newcommand{\pfrak}{\mathfrak{p}}
\newcommand{\qfrak}{\mathfrak{q}}
\DeclareMathOperator*{\colim}{colim}
\DeclareMathOperator{\im}{im}
\newcommand{\dcal}{\mathcal{D}}

\newcommand*{\xdash}[1][3em]{\rule[0.5ex]{#1}{0.55pt}}
\newcommand{\bcal}{\mathcal{B}}
\usepackage{dirtytalk}
\theoremstyle{definition}
\newtheorem{exercise}{Exercise}
\newtheorem*{exercise*}{Exercise}

\usepackage{enumitem}
\usepackage{mdframed}
\usepackage{hyperref}
\usepackage{csquotes}
\usepackage{hyperref}
\hypersetup{
    colorlinks,
    citecolor=black!50!green,
    filecolor=black,
    linkcolor=blue!50!black,
    urlcolor=black
}

\usepackage{graphicx}

\usepackage{tikz}
\usepackage[all]{xy}

\begin{document}

\begin{center}
\footnotesize{Álvaro Jiménez \hfill 2019/2020}
\line(1,0){474}\\[0.2in]
\textbf{\Large Homework 5. Advanced Algebra}\\[0.1in]
\large Advanced Algebra\\
\line(1,0){474}\\

\end{center}

\begin{exercise}
Let $\Gamma$ be a finite, connected graph with vertices $(e_1, \dots , e_E)$ and edges $(k_1, \dots , k_K)$. We endow the edges wih an orientation and obtain the incidence matrix $I_{\Gamma} \in \text{Mat}_{E \times K}(\mathbb{Z})$ of $\Gamma$ as 
$$
 (I_{\Gamma})_{ij} : =\begin{cases}
+1 & \text{if $k_j$ starts in $e_i$}\\
-1 & \text{if $k_j$ ends in $e_i$}\\
0 & \text{else}
\end{cases} \, .$$

Consider the complex $C_*$ given by 
$$C_1  = e_1\mathbb{Z} \oplus \cdots \oplus e_E \mathbb{Z} \, , \; C_0 = k \mathbb{Z} \oplus \cdots \oplus k_K \mathbb{Z} \, , \; C_n = 0$$
for all $n \geq 2$, endowed with the differential $d_1  \colon C_1 \to C_0$ given by multiplication with $I_{\Gamma}$ and $d_n = 0$ for $n \neq 1$. Prove the following statements: 
\begin{enumerate}[label = \alph*)]
    \item $(C_*, d)$ is a complex. 
    \item The only nonzero homology modules are $H_0(C_*)$ and $H_1(C_*)$. We have $\text{rk}(H_0(C_*)) = 0$ and $\text{rk}(H_1(C_*)) = E - K -1$. 
    \item The number of closed paths in $\Gamma$ is $E-K-1$. 
\end{enumerate}
\end{exercise}

\begin{solution}
At the moment I'm writing this there are some typos in the exercise. $(e_1, \dots , e_E)$ should be the \textbf{edges} and not the vertices, since we want $C_1$ to be the free group $\mathbb{Z}$-module generated by the edges and $C_0$ freely generated by the vertices. 

Also, the rank of $H_1$ is not $E - K -1$ but rather $E - K +1$. I'm changing the notation a little bit, so denote by $(e_1, \dots , e_E)$ the edges and by $(v_1, \dots , v_V)$ the vertices, so that $C_1$ is the free module generated by the edges and $C_0$ the free module generated by the vertices. 

\begin{enumerate}[label = \alph*)]
    \item Well, it is clearly a complex, since multiplication by $I_{\Gamma}$ is linear, and so it is a module homomorphism. Since there is only one non-trivial map, $d = d_1 \colon C_1 \to C_0$, we have that $d^2 = 0$, and so it is a complex. 
    \item Since the only non-trivial map is $d$, then the only non-trivial homology groups will be $H_0(C_*)$ and $H_1(C_*)$, which will be given respectively by 
    $$H_0 = C_0/\im d \, , \; H_1 = \ker d \, .$$
    
    Next, notice that if we have an edge $e$ joining two vertices $v_i$ and $v_j$, we can define a source and target map as 
    $$t(e) = v_j \, , s(e) = v_i \, .$$
    These induce then source and target maps $t, s \colon E \to V$ from the set of edges to the set of vertices, and we view mulitplication by the incidence matrix $I_{\Gamma}$ as $I_{\Gamma} e = t(e) - s(e)$. 
    
    Furthermore, if we are given a path $e_1, \dots , e_L$ from $v_i$ to $v_j$, we have 
    $$s(e_1) = v_i \, , \, t(e_L) = v_j \; \text{and} \; t(e_k) = s(e_{k+1}) \, ,$$
    so 
    $$I_{\Gamma} \cdot \sum_{k=1}^L e_k = \sum_{k=1}^L I_{\Gamma} \cdot e_k = \sum_{k=1}^L ( t(e_k) - s(e_k)) = v_j - v_i$$
    since it is a telescopic sum being $e_1, \dots , e_L$ a path. Therefore, we see that the elements $v_j - v_i$ are on the image of $d$ provided that there exists a path between them. But $\Gamma$ is path connected, which means that any two vertices have a path, so we can fix some vertex $\tilde{v}$, so that for any $v$ there exists a path from $v_j$ to $\tilde{v}$. This shows that $C_0$ is generated by 
    $$\{ \tilde{v}, v_1 - \tilde{v} , \dots , v_V - \tilde{v} \} \,$$
    and since the elements $v_1 - \tilde{v}$ belong to the image of $d$, they get zeroes out in $H^0$, so we only get that $\tilde{v}$ generates $H_0$, and so 
    $$\text{rk}(H_0(C_*)) = 1 \, .$$
    
    For $H_1$ we use the Rank-Nullity Theorem to see that 
    \begin{align*} 
    E = \text{rk}\, C_1 & = \text{rk} (\ker d) + \text{rk}(\im d)\\
    & = \text{rk}(H_1) + (\text{rk}(C_0) - \text{rk}(H_0))\\
    & = \text{rk}(H_1) + V - 1
    \end{align*}
    which shows that $\text{rk}(H_1(C_*)) = E - V + 1$. 
    \item A closed path is a sequence $e_1, \dots , e_L$ such that $s(e_1) = t(e_L)$. Then, we have $I_{\Gamma} \cdot \sum_{k=1}^L e_k = 0$, so these generate the kernel of $d$, and so there are $E -V +1$ of those, by the above. 
\end{enumerate}
\end{solution}


\begin{exercise}
\begin{enumerate}[label = \alph*)]
    \item Verify that 
    \begin{equation*}
        \begin{tikzcd}
        & \cdots \arrow[r, "d_{n+1}"] & C_n(K, R) \arrow[r, "d_n"] & \cdots \arrow[r, "d_2"] & C_1(K,R) \arrow[r, "d_1"] & C_0(K,R) \arrow[r] & 0
        \end{tikzcd}
    \end{equation*}
    is a complex. It is called the \textit{simplicial chain complex}. 
    \item Determine the simplicial chain complex with coefficients in $\mathbb{R}$ for the simplicial complex $K$ on $\{0, 1, 2, 3\}$ given  by all subsets of cardinality $\leq 2$. Compute the homology groups $H_n(K,\mathbb{R}) = \ker(d_n)/\im(d_{n+1})$. 
    \item For any (geometric) simplicial complex one can consruct and abstract simplicial complex by only retaining the set of vertices. Profe that, conversely, for any finite abstract simplicial complex $K$ one can construct a (geometric) simplicial complex $K'$. 
\end{enumerate}
\end{exercise}

\begin{solution}
\begin{enumerate}[label = \alph*)]
    \item First, we notice that the face maps $\partial_i \colon K_n \to K_{n-1}$ satisfy the identity 
    $$\partial_j \circ \partial_i = \partial_{i-1} \partial_j$$
    whenever $j < i$. Indeed, going by the left hand side we obtain the
    $$\{x_0, \dots , x_{j-1}, x_{j+1}, \dots , x_{i-1}, x_{i+1}, \dots , x_n\} \, .$$
    On the other hand, if we first apply $\partial_j$, then we would have shifted by 1 the index of $i$ so that $x_i$ is now in the position $i-1$, so we get 
    $$\{x_0, \dots , x_{j-1}, x_{j+1}, \dots , x_{i-1}, x_{i+1}, \dots , x_n\} $$
    as well. Then, we compute 
    \begin{align*}
        d \circ d & = d\left( \sum_{i=0}^{n} (-1)^i e_{\partial_i \sigma}\right) = \left( \sum_{j=0}^{n-1}(-1)^j e_{\partial_j}\right) \circ \left( \sum_{i=0}^{n} (-1)^i e_{\partial_i \sigma}\right)\\
        & = \sum_{0 \leq j < i \leq n}(-1)^{i+j} e_{\partial_j  \partial_i} + \sum_{0 \leq i \leq j \leq n-1}(-1)^{i+j}e_{\partial_j \partial_i \sigma} \\
        & = \sum_{0 \leq j < i \leq n} (-1)^{i+j}e_{\partial_{i-1} \partial_j} + \sum_{0 \leq i \leq j \leq n-1}(-1)^{i+j} e_{\partial_j \partial_i \sigma} \\
        & = \sum_{0 \leq i < j \leq n} (-1)^{i+j} e_{\partial_{j-1} \partial_i \sigma} + \sum_{0 \leq i \leq j \leq n-1}(-!)^{i+j}e_{\partial_j\partial_i \sigma}\\
        & = \sum_{0 \leq i \leq j \leq n-1}(-1)^{i+j-1}e_{\partial_j\partial_i \sigma} + \sum_{0 \leq i \leq j \leq n-1}(-1)^{i+j}e_{\partial_j\partial_i \sigma}\\
        & = 0
    \end{align*}
    where we have split the sum between $j <i$ and $i \leq j$, used the identity $\partial_j \circ \partial_i = \partial_{i-1} \partial_j$ from before and then identify $i$ with $j$ and finally $j$ with $j-1$. 
    \item We obtain the following complexes: 
    \begin{align*}
        C_0 & = \{ \{0\}, \{1\}, \{2\}, \{3\}\}\\
        C_1 & = \{ \{0,1\}, \{0,2\}, \{0,3\}, \{1,2\}, \{1,3\}, \{2,3\}\}
    \end{align*}
    and the rest of them all trivial. 
    For the zero-th homology group $H_0(K,\mathbb{R})$, we need to look at $\mathrm{im}(d_1)$. This is given by 
    \begin{align*}
        d_1(\{0,1\}) &= \{1\}-\{0\}\\
        d_1(\{0,2\}) & =\{2\} -\{0\}\\
        d_1(\{0,3\}) & = \{3\} - \{0\}\\
        d_1(\{1,2\}) & = \{2\} - \{1\}\\
        d_1(\{1,3\}) & = \{3\}-\{1\}\\
        d_1(\{2,3\}) & = \{3\} - \{2\}
    \end{align*}
    which we can write in matrix form as 
    $$\mathrm{im}(d_1) = 
    \begin{pmatrix}
    -1 & 1 & 0 & 0\\
    -1 & 0 & 1 & 0\\
    -1 & 0 & 0 & 1\\
    0 & -1 & 1 & 0\\
    0 & -1 & 0 & 1\\
    0 & 0 & -1 & 1
    \end{pmatrix}$$
    which has rank 3. This $\mathrm{im}(d_1) \cong \mathbb{R}^3$ and thus 
    $$H_0(K, \mathbb{R}) = C_0/\mathrm{im}(d_1) \cong \mathbb{R}^4/\mathbb{R}^3 \cong \mathbb{R} \, .$$
    
    For the other one, however, now we have that $\mathrm{im}(d_2) \cong 0$, and thus 
    $$H_1(K, \mathbb{R}) \cong \ker(d_1)/\mathrm{im}(d_2) \cong \mathbb{R}^3/0 \cong \mathbb{R}^3 \, .$$
    
    The rest of the homology groups are trivial. 
    \item Given an anstract simplicial complex $K$, define a category $\text{Face}(K)$ whose objects are the faces of $K$ and morphisms are given by inclusions. Let 
    $$|\Delta^n| = \left\{ (t_0, \dots, t_n) \in \mathbb{R}^{n+1} \mid \sum_{i=0}^n t_i = 1, t_i \geq 0 \right\}$$
    be the standard topological $n$-simplex. Define a functor 
    \begin{align*}
        \fcal \colon \text{Face}(K) & \to \textbf{Top}\\
        X & \mapsto |\Delta^{\dim X}| \, .
    \end{align*}
    Given another face $Y$ of dimension $n \leq m = \dim X$, there is a map $F(Y) \to F(X)$ consisting of sending $|\Delta^m|$ to the $m$-dimensional face of $|\Delta^n|$. Then we define a simplicial complex as the colimit of the functor $\fcal$. 
\end{enumerate}
\end{solution}


\begin{exercise}
Let $M, N$ be smooth manifolds. 
\begin{enumerate}[label = (\alph*)]
    \item Show that the dimension of the 0-th de Tham cohomology group $H^0_{dR}(M)$ euals the number of connected components of $M$. 
    \item Let $f, g \colon M \to N$ be homotopic smooth maps. Prove that hte induced maps $\overline{f}, \overline{g} \colon \Omega^*(N) \to \Omega^*(M)$ of complexes (on $\Omega^p(N)$ they are given by the pullbacks $f^*, g^*$) are homotopic. In particular, we have $H_{dR}^p(f) = H^p_{dR}(g)$ for all $p \geq 0$. 
    \item Compute the de Rham cohomology groups for the $n$-dimensional sphere $S^n$. 
    \item Let $v, w \in \mathbb{R}^n$. Compute the de Rham cohomology groups for $\mathbb{R}^n \setminus \{v\}$ and $\mathbb{R}^n \setminus \{v, w\}$. 
\end{enumerate}
\end{exercise}

\begin{solution}
\begin{enumerate}[label = \alph*)]
    \item We can decompose $M$ as a sum of its connected components, and since cohomology commutes with finite coproducts, we have that 
    $$H^0_{dR}(M) \cong \bigoplus_{i = 1}^n H^0_{dR}(M_i)$$
    where $M_i$ denotes the $i$-th connected component of $M$. Now, $H^0_{dR}(M_i)$ is simply the set of closed 0-forms on $M_i$, that is, smooth functions on $M_i$ such that $d f = 0$. These is the group of locally constant functions on $M_i$, but since $M_i$ is connected, locally constant implies globally constant, so $H^0_{dR}(M_i) \cong \mathbb{R}$. Therefore, 
    $$H^0_{dR}(M) \cong \mathbb{R}^n$$
    so the dimension of the 0-th de Rham cohomology group equals the number of conncted components. 
    \item Let $H \colon M \times [0,1] \to N$ be a homotopy between $f$ and $g$, that is, such that $H(x,0) = f(x)$ and $H(x,1) = g(x)$. Consider a cocycle $\omega \in \Omega^k(N)$. The pullback is then along $H$ can be wrriten as 
    $$H^* \omega = \omega_0 + \mathrm{d}\, t \land \omega_1$$
    where $\omega_0 \in \Omega^k(M)$ and $\omega_1 \in \Omega^{k-1}(M)$, so we get
    $$f^*\omega = \omega_0 = g^* \omega \, .$$
    Since $F^*\omega$ is a cocylce, then 
    $$0 = \mathrm{d}\, F^* \omega = \mathrm{d} \, t \land \left( \frac{\partial \omega_0}{\partial t} - d_M \omega_1 \right) + \cdots $$
    and thus 
    $$f^*\omega - g^*\omega = \omega_0(1) - \omega_0(0) = \int_0^1 \frac{\partial \omega_0}{\partial t} \, \mathrm{d}\, t = \int_0^1 d_M \omega_1 \, \mathrm{d}\, t = d_M \int_0^1 \omega_1\, \mathrm{d}\, t \, .$$
    
    This means that $f^*\omega$ and $g^*\omega$ are the same in cohomology. 
    \item First we need the case base for $S^1$. We do this by choosing a cover $S^1 = U \cup V$ of two semicircles overlapping, so that $U \cap V$ is the union of two disconnected segments. Applying Mayer-Vietoris to this cover, it becomes 
    \begin{equation*}
        \begin{tikzcd}
        & 0 \arrow[r] & \mathbb{R} \arrow[r] & \mathbb{R}^2 \arrow[r]& \mathbb{R}^2 \arrow[r] & H^1(S^1) \arrow[r] & 0 
        \end{tikzcd}
    \end{equation*}
    since $H^0(S^1) \cong \mathbb{R}$, and $H^0(U) \cong \mathbb{R} \cong \mathbb{R}^2$. This implies that $H^1(S^1) = \mathbb{R}$. 
    The claim is that 
    $$H^n(S^m) = 
    \begin{cases}
    \mathbb{R} & \text{if $n = 0, m$}\\
    0 & \text{otherwise}
    \end{cases}$$
    To see this. choose a cover of $S^m$ by $U$ and $V$ being $S^m$ minus the north and south pole respectively. Using Poincare Lemma we have that 
    $$H^k(U) = H^k(V) \cong 0$$
    for all $k$. On the other hand, $U \cap V$ deformation retracts onto $S^{m-1}$, for which the induction hypothesis applies. Hence we know all cohomology groups except $H^n(S^m)$ for $1 \leq n \leq m$. 
    
    If $n = 1$, first notice that the map $H^0(S^m) \to H^0(U) \oplus H^0(V)$ has trivial kernel and its image is isomorphic to $\mathbb{R}$, so $H^0(S^n) \cong \mathbb{R}$. On the other hand, the connecting homomorphism $\delta \colon H^0(S^{m-1}) \to H^1(S^m)$ is surjective since the map $H^1(S^2) \to H^1(U) \oplus H^1(V)$ is trivial. Using Poincaré Lemma for $H^k(U) = H^k(V) = 0$ for $k >0$, a similar argument as used for $H
   ^1(S^1)$ shows that $H^1(S^m) = 0$.  If $1 < n < m$, then all the mapso going from $H^n(S^m)$ and into it are trivial, so $H^n(S^m) = 0$. 
   
   Finally, if $m = n$, the connecting homomorphism $\delta$ is surjective, and $H^{n-1}(S^{n-1}) \cong \mathbb{R}$ by induction hypothesis. Moreover, the image of the substraction map that goes into $H^{n-1}(S^{n-1})$ is zero, so the last connecting homomorphism $\delta$ has trivial kernel and is therefore an isomorphism. Hence $H^n(S^n) \cong \mathbb{R}$. 
   \item We have done the first one. $\mathbb{R} \setminus \{v\}$ deformation retracts onto the sphere $S^{n-1}$, for which we already know the cohomology groups. 
   
   For $\mathbb{R}^n \setminus \{u, v\}$, I'm guessing that this deformation retracts onto the wedge of two $S^{n-1}$, so we would have 
   $$H^m(\mathbb{R} \setminus \{u, v\}) \cong H^m(S^{n-1}) \oplus H^m(S^{n-1}) \, .$$
\end{enumerate}
\end{solution}



\end{document}
